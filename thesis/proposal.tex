\documentclass[a4paper,10pt]{article}
\usepackage[utf8]{inputenc}
\usepackage[colorlinks, allcolors=blue]{hyperref}
\bibliographystyle{apalike}

%opening
\title{Fuzzy land cover classification using PROBA-V satellite data}
\author{Dainius Masili\=unas}

\begin{document}

\maketitle

\begin{abstract}

\end{abstract}

\section{Introduction}

While classification uncertainty is not a direct measure of class membership, they are nonetheless correlated. As such, it has been used by numerous authors as a proxy indicator of class membership, achieving satisfactory classification accuracy \cite{foody2002accuracy}.

For hard classification, accuracy assessment is rather straight-forward: typically a confusion matrix is employed for this purpose, showing how many pixels in the image have been classified correctly, and how many incorrectly. Such a matrix makes it simple to tell which classes are hard to discern from one another, as well as allows for deriving statistics such as users' accuracy, producers' accuracy, and total accuracy. Unfortunately, a standard confusion matrix is not applicable in the context of fuzzy classification, since misclassification in this case is not absolute, but rather a matter of degree \cite{foody2002accuracy}.

\bibliography{bibliography}

\end{document}
